\documentclass{article}
\usepackage{array}
\usepackage{multirow}

\begin{document}

% 単純な表
\begin{table}[htbp]
    \centering
    \begin{tabular}{|c|c|c|}
        \hline
        Header 1 & Header 2 & Header 3 \\
        \hline
        Row 1, Col 1 & Row 1, Col 2 & Row 1, Col 3 \\
        \hline
        Row 2, Col 1 & Row 2, Col 2 & Row 2, Col 3 \\
        \hline
        Row 3, Col 1 & Row 3, Col 2 & Row 3, Col 3 \\
        \hline
    \end{tabular}
    \caption{Basic Table Example}
    \label{tab:basic_table}
\end{table}

% 複数列を結合した表
\begin{table}[htbp]
    \centering
    \begin{tabular}{|c|c|c|}
        \hline
        \multicolumn{2}{|c|}{Merged Columns} & Header 3 \\
        \hline
        Row 1, Col 1 & Row 1, Col 2 & Row 1, Col 3 \\
        \hline
        Row 2, Col 1 & \multicolumn{2}{c|}{Merged Columns across Col 2 and Col 3} \\
        \hline
        Row 3, Col 1 & Row 3, Col 2 & Row 3, Col 3 \\
        \hline
    \end{tabular}
    \caption{Table with \texttt{\textbackslash multicolumn}}
    \label{tab:multicolumn_table}
\end{table}

% 複数の行を結合した表
\begin{table}[htbp]
    \centering
    \begin{tabular}{|c|c|c|}
        \hline
        \multirow{3}{*}{Multirow} & Col 2, Row 1 & Col 3, Row 1 \\
        \cline{2-3}
        & Col 2, Row 2 & Col 3, Row 2 \\
        \cline{2-3}
        & Col 2, Row 3 & Col 3, Row 3 \\
        \hline
    \end{tabular}
    \caption{Table with \texttt{\textbackslash multirow}}
    \label{tab:multirow_table}
\end{table}

\end{document}
